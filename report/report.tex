\documentclass[12pt,english]{article}
\usepackage[T1]{fontenc}
\usepackage{babel}
\usepackage[ampersand]{easylist}
\usepackage{scrextend}
\usepackage[utf8]{inputenc}

\begin{document}

%%%%%%%%%%%%%%%%%%%%%%%%%%%%%%%%%%%%%%%%%%%%%%%%%%%%%%%%%%%%%%%%%%%%%%%%%%%%%%%%
% Document Setup

\pagenumbering{arabic}

%%%%%%%%%%%%%%%%%%%%%%%%%%%%%%%%%%%%%%%%%%%%%%%%%%%%%%%%%%%%%%%%%%%%%%%%%%%%%%%%
% Title block - Page 0

% Cover page. Clearly indicate the names of the team members.
% Also provide a table comparing the given specs and achieved spec (remember
% this is for the NON-BONUS part of the project only). You also need to provide
% us a percentage-wise break down of area consumed by different sections of
% bias circuit (VNMOS-bias, VPMOS-bias & bias generator circuit) as compared to
% your core circuit area (refer to Figure 3). For area calculation, use
% AMOS = W * L. Also include the area of resistors in your design. Assume that
% the sheet resistance is 5.5 kohm/sq and the minimum length of resistor is
% 1 um

\author{
  Kankanala, Usha\\
  \texttt{ukankana@stanford.edu}\\
  \texttt{SUID:xxxx}
  \and
  Lenius, Samuel\\
  \texttt{lenius@stanford.com}\\
  \texttt{SUID:06091240}
}

\title{2015 EE214A Design Project}

\maketitle

%%%%%%%%%%%%%%%%%%%%%%%%%%%%%%%%%%%%%%%%%%%%%%%%%%%%%%%%%%%%%%%%%%%%%%%%%%%%%%%%
% Performance Summary Block - Page 0

\begin{abstract}
  Area Breakdown:
  \begin{addmargin}[1em]{0em}
  \begin{itemize}
    \item PMOS Bias Gen: 25\%
    \item NMOS Bias Gen: 25\%
    \item Core circuit: 25\%
    \item Resistors: 25\%
  \end{itemize}
  \end{addmargin}
\end{abstract}

\pagebreak

%%%%%%%%%%%%%%%%%%%%%%%%%%%%%%%%%%%%%%%%%%%%%%%%%%%%%%%%%%%%%%%%%%%%%%%%%%%%%%%%
% Design Outline - Page 1

% Outline of your design. How did you approach this problem? What are some of
% your key design choices? Flow charts and graphs of how the trade-offs are
% connected provide the best clarity in explaining (we'll show some examples in
% class). Half of the grading (i.e. 25 Points) is related to Design Flow,
% Insight and Optimization Strategy. The clarity of your discussion and the
% insight you give, starting on Page 1, is a major factor in doing well for
% these 25 Points.

\section{Design Outline}
Text of design outline

\pagebreak

%%%%%%%%%%%%%%%%%%%%%%%%%%%%%%%%%%%%%%%%%%%%%%%%%%%%%%%%%%%%%%%%%%%%%%%%%%%%%%%%
% Design Schematic - Page 2

% Schematic diagram of your final design, with component values (i.e. W, L
% values etc.), node voltages and bias currents (from SPICE .op simulation)
% clearly labeled (Spend time to make this schematic complete, readable, and
% clear!). Show component values right next to the components, and currents next
% to the branches (i.e., absolutely, positively do not make us refer to a
% look-up table!). Annotate all transistors with their drain current, gate
% overdrive VOV (from SPICE .op simulation) and W/L.

\section{Design Schematic}
Text of design schematic

\pagebreak

%%%%%%%%%%%%%%%%%%%%%%%%%%%%%%%%%%%%%%%%%%%%%%%%%%%%%%%%%%%%%%%%%%%%%%%%%%%%%%%%
% Calculation of Key Design Parameters - Page 3-6

% Calculation of key design parameters, such as transconductances, bias
% currents, etc. This is the most important section of your report for giving
% critical discussion! Compare the most relevant hand calculated values with
% final SPICE values in a table and discuss discrepancies (percentage
% differences will be clear, you need to show you understand them). Make sure to
% include the total power dissipation of your design (calculated value and SPICE
% result). The lowest power designs will not automatically score the highest
% grades. The methodology you used to justify your design choices and component
% values is far more important (see section on point distribution below).

\section{Calculation of Key Design Parameters}
Text of calculating key design parameters

\pagebreak

%%%%%%%%%%%%%%%%%%%%%%%%%%%%%%%%%%%%%%%%%%%%%%%%%%%%%%%%%%%%%%%%%%%%%%%%%%%%%%%%
% Bode Diagrams - Page 7

% Simulated Bode Plots of A(jw), magnitude and phase. Clearly annotate the
% achieved gain and bandwidth. Annotate your hand-calculated values in the same
% plots, noting any specific features of interest (either from the results
% themselves or based on what you've learned in hand calculations or scripting
% the design). Plots must be annotated with meaningful comments/observations.

\section{Simulated Bode Plots}
Text of bode plots

\pagebreak

%%%%%%%%%%%%%%%%%%%%%%%%%%%%%%%%%%%%%%%%%%%%%%%%%%%%%%%%%%%%%%%%%%%%%%%%%%%%%%%%
% Transient Response - Page 8

% Show a transient simulation plot of the output for a 1 MHz, 1 µA sinusoida
% input current. Make sure that there is no distortion.

\section{Simulated Transient Response}
Text of transient response

\pagebreak

%%%%%%%%%%%%%%%%%%%%%%%%%%%%%%%%%%%%%%%%%%%%%%%%%%%%%%%%%%%%%%%%%%%%%%%%%%%%%%%%
% Comments and Conclusions - Page 9

% Comments and conclusion. Here, you can convey issues you may have had, or
% things you have learned/not learned in this project.

\section{Comments and Conclusion}
Write your conclusion here.

\pagebreak

%%%%%%%%%%%%%%%%%%%%%%%%%%%%%%%%%%%%%%%%%%%%%%%%%%%%%%%%%%%%%%%%%%%%%%%%%%%%%%%%
% Spice Netlist - Appendix I

% Final SPICE netlist and .op output. Include only the MOSFET and node voltage
% listing from the .op output.

\section{Appendix I}
Netlist here

%%%%%%%%%%%%%%%%%%%%%%%%%%%%%%%%%%%%%%%%%%%%%%%%%%%%%%%%%%%%%%%%%%%%%%%%%%%%%%%%
% Bonus Problem - Appendix II - 5pg max

% If you attempt the bonus problem, describe the limitations of the original
% circuit and your proposed changes to the architecture of this circuit. Draw a
% schematic diagram of your final design with new architecture (follow
% instructions similar to those for page 2) and explain how it helps you achieve
% the new bonus specs. Include the simulated bode plots of A(jw), magnitude and
% phase, annotating the new gain and bandwidth you achieved. On the final page,
% include a table comparing the given specs, the achieved specs using given
% architecture, and the achieved specs for your new architecture.

\section{Appendix II}
Bonus problem here


\end{document}
